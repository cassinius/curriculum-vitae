%%%%%%%%%%%%%%%%%%%%%%%%%%%%%%%%%%%%%%%%%
% Twenty Seconds Resume/CV
% LaTeX Template
% Version 1.1 (8/1/17)
%
% This template has been downloaded from:
% http://www.LaTeXTemplates.com
%
% Original author:
% Carmine Spagnuolo (cspagnuolo@unisa.it) with major modifications by 
% Vel (vel@LaTeXTemplates.com)
%
% License:
% The MIT License (see included LICENSE file)
%
%%%%%%%%%%%%%%%%%%%%%%%%%%%%%%%%%%%%%%%%%

%----------------------------------------------------------------------------------------
%	PACKAGES AND OTHER DOCUMENT CONFIGURATIONS
%----------------------------------------------------------------------------------------

\documentclass[a4paper]{twentysecondcv} % a4paper for A4

%\usepackage{natbib}
\usepackage[natbibapa]{apacite}
\usepackage{hyperref}

%----------------------------------------------------------------------------------------
%	 PERSONAL INFORMATION
%----------------------------------------------------------------------------------------

% If you don't need one or more of the below, just remove the content leaving the command, e.g. \cvnumberphone{}

\profilepic{./images/bernd_2019_haircut} % Profile picture

\cvname{Bernd Malle} % Your name
\cvjobtitle{SW Developer} % Job title/career

\cvdate{8 October 1979} % Date of birth
\cvaddress{Graz, Austria} % Short address/location, use \newline if more than 1 line is required
\cvnumberphone{+43 678/12-75-414} % Phone number
\cvsite{http://berndmalle.com} % Personal website
\cvmail{bernd@inodis.net} % Email address

%----------------------------------------------------------------------------------------

\begin{document}

%----------------------------------------------------------------------------------------
%	 ABOUT ME
%----------------------------------------------------------------------------------------


\aboutme{
	Bernd is an enthusiastic Software guy interested in combining client-side development with algorithmic problem solving. 
	
	\vspace{2mm}
	
	He is currently applying his skills to creating iNodis, a next generation online recommender service via personalized, context-aware graphs directly on the client device.
	
	\vspace{2mm}

	Although Bernd left his academic work for entrepreneurial challenges, he retained a scientific mind-set and sound, experimental approach.
} % To have no About Me section, just remove all the text and leave \aboutme{}

%----------------------------------------------------------------------------------------
%	 SKILLS
%----------------------------------------------------------------------------------------

% Skill bar section, each skill must have a value between 0 an 6 (float)
\skills{
	{negotiation \& PR/3.6},
	{web development/4},
	{software architecture/4.4},
	{graph implementation \& testing/4.4},
	{presentation/4.8},
	{organizational mindset/5.2},
	{JS algorithmic development/5.6},
	{evolutionary thinking/6}}
	

%------------------------------------------------

% Skill text section, each skill must have a value between 0 an 6
% \skillstext{{lovely/4},{narcissistic/3}}

%----------------------------------------------------------------------------------------

\makeprofile % Print the sidebar

%----------------------------------------------------------------------------------------
%	 INTERESTS
%----------------------------------------------------------------------------------------

\section{Core beliefs \& interests}

\begin{itemize}	
	\item \textbf{Client-side ML}. Highly-centralized data infrastructures will subside to swarms of client nodes predicting on overlapping knowledge bases (\emph{"local spheres"}), exchanging insights about their conclusions when needed.
	
	\item \textbf{Scalability \& Privacy}. CS-ML will enable us to build networks that scale at almost no cost to the provider \& make use of highly-personalized data without transmitting them, thus solving the KI vs. Privacy conundrum.
	
	\item \textbf{Context over mass}. Modern AI is hopelessly incapable of understanding simple snippets like \emph{"At the airport. Will be back in 2 weeks"} while knowing the context (sender is \emph{wife} vs. \emph{boss}) triggers a slew of useful inferential data. Thus millions of samples can be replaced by a small contextual knowledge graph.
	
	\item \textbf{Distributed startup}. Given the above ideas \& a micro-service oriented architecture, the next generation of startups will be able to operate in very small teams distributed globally, with independent development cycles.
	
	\item \textbf{Current software interests}. Front-end frameworks, data-driven SVGs, compiling from server-side languages (C++, Rust) to Webassembly.
	
	\item \textbf{Current theoretical interests}. Graph theory (partitioning, parallel centralities), embeddings (words, graphs, everything), context-based ML, split testing \& metrics-driven development.
	
	% \item \textbf{Entrepreneurial interests}. Low-cost/investment startups 
\end{itemize}

% \subsection{Interests}


%----------------------------------------------------------------------------------------
%	 EXPERIENCE
%----------------------------------------------------------------------------------------

\section{Experience}

\begin{twenty} % Environment for a list with descriptions
	\twentyitemshort{2018-->}{Founded iNodis, joined \emph{UT11.net} as `Entrepreneur in Residence`}
	\twentyitemshort{2016-2018}{Member of the Austrian KIRAS funded project \emph{Darknet}
	} %{Project}{}
	\twentyitemshort{2015-2018}{IT Security Researcher @ \emph{(SBA) Research GmbH}, Vienna, Austria} %{Professional}{}	
	\twentyitemshort{2016-2017}{Supervised 2 BSc. theses and 1 MSc. project} %{Supervision}{}
	\twentyitemshort{2015-2016}{Project \emph{Graphinius}: An interactive graph exploration platform} %{Project}{}	
	\twentyitemshort{2014-2015}{Project \emph{iKnodis}: Graph extraction from medical images} %{Project}{}	
	\twentyitemshort{2014-2018}{Member of the \emph{HCI-KDD} research group at Medical University Graz} %{Research}{}
	\twentyitemshort{2009-2014}{Independent Software / Web developer on a contractual basis} %{Professional}{}
	\twentyitemshort{2008-2009}{Programmer at \emph{Siemens Medical}, Graz, Austria.
	} %{Professional}{}

	%\twentyitem{<dates>}{<title>}{<location>}{<description>}
\end{twenty}


%----------------------------------------------------------------------------------------
%	 EDUCATION
%----------------------------------------------------------------------------------------

\section{Education}

\begin{twenty} % Environment for a list with descriptions
	\twentyitem{2016-2018}{Ph.D. {\normalfont candidate in Computer Science}}{TU Graz}{\emph{Research in privacy-aware Machine Learning}, ongoing}
	\twentyitem{2014-2016}{M.Sc. software engineering }{TU Graz}{Thesis in \emph{Graphinius - an online graph exploration platform}}
	\twentyitem{2005-2014}{B.Sc. software engineering \& business administration}{TU Graz}{Thesis in \emph{graph extraction from image data}}
	\twentyitem{2002-2005}{Studies of Economics}{KF University, Graz}{Special interest in political economics; not graduated}
	\twentyitem{1999-2002}{Abendmatura @ College of further education}{Graz, Austria}{Specializing in mathematics and physics.}
	\twentyitem{1997-1998}{ITCP Information Technology Certified Professional}{Wifi Graz}{}
	%\twentyitem{<dates>}{<title>}{<location>}{<description>}
\end{twenty}


%----------------------------------------------------------------------------------------
%	 PUBLICATIONS
%----------------------------------------------------------------------------------------
\vspace{-0.5cm}

\section{Publications (first author)}

\begin{twentyshort} % Environment for a short list with no descriptions
	\twentyitemshort{2018}{\emph{The Need for Speed.} Comparison of JS vs. C++ -> (W)ASM} %{arXiv}{\url{https://arxiv.org/pdf/1802.03707.pdf}}
	\twentyitemshort{2017}{\emph{Interactive Anonymization for Privacy aware ML}} %{IAL@ECML}{\url{https://core.ac.uk/download/pdf/154340412.pdf}}
	\twentyitemshort{2017}{\emph{The more the merrier} - federated ML from local spheres} %{LNCS}{\url{https://link.springer.com/chapter/10.1007/978-3-319-66808-6_24}}
	\twentyitemshort{2017}{\emph{Do not disturb?} - classifier behavior on perturbed datasets} %{LNCS}{\url{https://link.springer.com/chapter/10.1007/978-3-319-66808-6_11}}
	\twentyitemshort{2016}{\emph{The Right to be forgotten.} ML on perturbed knowledge bases} %{LNCS}{\url{https://link.springer.com/chapter/10.1007/978-3-319-45507-5_17}}
	
	%\twentyitemshort{<dates>}{<title/description>}
\end{twentyshort}


%----------------------------------------------------------------------------------------
%	 CONFERENCES
%----------------------------------------------------------------------------------------

\section{Conference talks}

\begin{twentyshort} % Environment for a short list with no descriptions
	\twentyitemshort{2017}{ECML - European Conf. on Machine Learning, Skopje, Macedonia}
	\twentyitemshort{2017}{CD-MAKE - ML and Knowledge Extraction, Reggio Calabria, Italy}
	\twentyitemshort{2017}{ARES - Availability, Reliability \& Security, Reggio Calabria, Italy}
	\twentyitemshort{2017}{Security Forum Hagenberg, Linz, Austria}	
	\twentyitemshort{2016}{ÖGAI (Austrian society for AI) Meeting , Klagenfurt, Austria}
	\twentyitemshort{2016}{ARES - Availability, Reliability \& Security, Salzburg, Austria}
	%\twentyitemshort{<dates>}{<title/description>}
\end{twentyshort}


%----------------------------------------------------------------------------------------
%	 SECOND PAGE EXAMPLE
%----------------------------------------------------------------------------------------

% \newpage % Start a new page

% \makeprofile % Print the sidebar

%----------------------------------------------------------------------------------------
%	 OTHER INFORMATION
%----------------------------------------------------------------------------------------

% \section{Other information}

% \subsection{Review}

% Bernd graduated in Software Development with a MSc. Thesis on Graphinius – an online graph exploration and analysis platform at Graz University of Technology, supervised by Prof. Dr. A. Holzinger.
% He participated in 2 software-related research projects so far (one of which he initiated) and has contributed Web based software for graph extraction from (skin cancer) images as well as a generic graph computation and analysis library for real-time computations inside a Browser. This work was supplemented by his commitment to supervise a Masters project in web-based 3D graph visualization as well as a Bachelor thesis in applying Graphinius to the simulation of neurological processes. Bernd also recognizes the importance of human-in-the-loop approaches to data mining tasks and is therefore concerned about finding ways to visualize high-dimensional data structures for intuitive user understanding, currently within research experiments about interactive anonymization. Rounding off his academic curiosity, Bernd is always interested in packaging his insights into usable, web-based, Software.


%----------------------------------------------------------------------------------------

\bibliographystyle{apacite}
% \bibliography{references}

\end{document} 

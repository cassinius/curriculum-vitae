%%%%%%%%%%%%%%%%%%%%%%%%%%%%%%%%%%%%%%%%%
% Twenty Seconds Resume/CV
% LaTeX Template
% Version 1.1 (8/1/17)
%
% This template has been downloaded from:
% http://www.LaTeXTemplates.com
%
% Original author:
% Carmine Spagnuolo (cspagnuolo@unisa.it) with major modifications by 
% Vel (vel@LaTeXTemplates.com)
%
% License:
% The MIT License (see included LICENSE file)
%
%%%%%%%%%%%%%%%%%%%%%%%%%%%%%%%%%%%%%%%%%

%----------------------------------------------------------------------------------------
%	PACKAGES AND OTHER DOCUMENT CONFIGURATIONS
%----------------------------------------------------------------------------------------

\documentclass[a4paper]{twentysecondcv} % a4paper for A4

%\usepackage{natbib}
\usepackage[natbibapa]{apacite}

%----------------------------------------------------------------------------------------
%	 PERSONAL INFORMATION
%----------------------------------------------------------------------------------------

% If you don't need one or more of the below, just remove the content leaving the command, e.g. \cvnumberphone{}

\profilepic{./images/bernd} % Profile picture

\cvname{Bernd Malle} % Your name
\cvjobtitle{SW Developer} % Job title/career

\cvdate{8 October 1979} % Date of birth
\cvaddress{Graz, Austria} % Short address/location, use \newline if more than 1 line is required
\cvnumberphone{+43 6781275414} % Phone number
\cvsite{http://berndmalle.com} % Personal website
\cvmail{bernd.malle@gmail.com} % Email address

%----------------------------------------------------------------------------------------

\begin{document}

%----------------------------------------------------------------------------------------
%	 ABOUT ME
%----------------------------------------------------------------------------------------

\aboutme{
	Bernd is an enthusiastic Software guy interested in combining client-side (Web) development with algorithmic problem solving. He is currently applying his skills to creating Lemontiger, a next-generation team collaboration tool fueled by a graph-based resource recommender. 
	Although Bernd left his academic work for entrepreneurial challenges, he retained a scientific mind-set and sound, experimental approach.
} % To have no About Me section, just remove all the text and leave \aboutme{}

%----------------------------------------------------------------------------------------
%	 SKILLS
%----------------------------------------------------------------------------------------

% Skill bar section, each skill must have a value between 0 an 6 (float)
\skills{
	{Think of more/2.0},
	{Python/3.5},
	{Graph Theory/4.0},
	{Lean thinking/4.4},
	{SW Architecture/4.8},
	{Java- \& Typescript/5.2},
	{client-side development/5.7}}

%------------------------------------------------

% Skill text section, each skill must have a value between 0 an 6
% \skillstext{{lovely/4},{narcissistic/3}}

%----------------------------------------------------------------------------------------

\makeprofile % Print the sidebar

%----------------------------------------------------------------------------------------
%	 INTERESTS
%----------------------------------------------------------------------------------------

\section{Interests}



%----------------------------------------------------------------------------------------
%	 EDUCATION
%----------------------------------------------------------------------------------------

\section{Education}

\begin{twenty} % Environment for a list with descriptions
	\twentyitem{2016-2018}{Ph.D. {\normalfont candidate in Computer Science}}{TU Graz}{\emph{Research in privacy-aware Machine Learning}, ongoing}
	\twentyitem{2014-2016}{M.Sc. software engineering }{TU Graz}{Thesis in \emph{Graphinius - an online graph exploration platform}}
	\twentyitem{2005-2014}{B.Sc. software engineering \& business administration}{TU Graz}{Thesis in \emph{graph extraction from image data}}
	\twentyitem{2002-2005}{Studies of Economics}{KF University, Graz}{Special interest in political economics; not graduated}
	\twentyitem{1999-2002}{Abendmatura @ College of further education}{Graz, Austria}{Specializing in mathematics and physics.}
	\twentyitem{1997-1998}{ITCP Information Technology Certified Professional}{Wifi Graz}{}
	%\twentyitem{<dates>}{<title>}{<location>}{<description>}
\end{twenty}

%----------------------------------------------------------------------------------------
%	 PUBLICATIONS
%----------------------------------------------------------------------------------------

\section{Publications}

\begin{twentyshort} % Environment for a short list with no descriptions
	\twentyitemshort{2018}{\citeyearpar{Need4Speed} The Need for Speed. Comparison of JS vs. C++ - generated ASM.js / WASM}
	\twentyitemshort{2017}{\citeyearpar{InteractiveAnonymization} Interactive Anonymization for Privacy aware Machine Learning}
	\twentyitemshort{2017}{\citeyearpar{FedPaper} The more the merrier - federated learning from local sphere recommendations}
	\twentyitemshort{2017}{\citeyearpar{PAML2} Do not disturb? - classifier behavior on perturbed datasets}
	\twentyitemshort{2016}{\citeyearpar{PAML1} The Right to be forgotten. Towards Machine Learning on perturbed knowledge bases}
	
%	\cite{PAML1}
%	\cite{Need4Speed}
	
	%\twentyitemshort{<dates>}{<title/description>}
\end{twentyshort}

%----------------------------------------------------------------------------------------
%	 CONFERENCES
%----------------------------------------------------------------------------------------

\section{Conference talks}

\begin{twentyshort} % Environment for a short list with no descriptions
	\twentyitemshort{2017}{ECML - European Conference on Machine Learning, Skopje, Macedonia}
	\twentyitemshort{2017}{CD-MAKE - Cross disciplinary ML and Knowledge Extraction, Reggio Calabria, Italy}
	\twentyitemshort{2017}{ARES - Conference on Availability, Reliability and Security, Reggio Calabria, Italy}
	\twentyitemshort{2017}{Security Forum Hagenberg, Linz, Austria}	
	\twentyitemshort{2016}{ÖGAI (Austrian society for AI) Meeting , Klagenfurt, Austria}
	\twentyitemshort{2016}{ARES - Conference on Availability, Reliability and Security, Salzburg, Austria}
	%\twentyitemshort{<dates>}{<title/description>}
\end{twentyshort}

%----------------------------------------------------------------------------------------
%	 EXPERIENCE
%----------------------------------------------------------------------------------------

\section{Experience}

\begin{twenty} % Environment for a list with descriptions
	\twentyitem{2016-2018}{Member of the Austrian KIRAS funded project Darknet.
	}{Project}{}

	\twentyitem{2015-2018}{Employee at Secure Business Austria (SBA) Research GmbH, Vienna, Austria}{Job}{}
	
	\twentyitem{2016-2017}{Supervised 2 BSc. theses and 1 MSc. project}{Supervision}{}

	\twentyitem{2015-2016}{Initiated Project Graphinius: An online graph exploration and analysis platform}{Project}{}
	
	\twentyitem{2014-2018}{Member of the HCI-KDD research group at Medical University Graz, Austria.
	}{Research}{}
	
	%\twentyitem{<dates>}{<title>}{<location>}{<description>}
\end{twenty}

%----------------------------------------------------------------------------------------
%	 SECOND PAGE EXAMPLE
%----------------------------------------------------------------------------------------

\newpage % Start a new page

\makeprofile % Print the sidebar

%----------------------------------------------------------------------------------------
%	 OTHER INFORMATION
%----------------------------------------------------------------------------------------

\section{Other information}

\subsection{Review}

Bernd graduated in Software Development with a MSc. Thesis on Graphinius – an online graph exploration and analysis platform at Graz University of Technology, supervised by Prof. Dr. A. Holzinger.
He participated in 2 software-related research projects so far (one of which he initiated) and has contributed Web based software for graph extraction from (skin cancer) images as well as a generic graph computation and analysis library for real-time computations inside a Browser. This work was supplemented by his commitment to supervise a Masters project in web-based 3D graph visualization as well as a Bachelor thesis in applying Graphinius to the simulation of neurological processes. Bernd also recognizes the importance of human-in-the-loop approaches to data mining tasks and is therefore concerned about finding ways to visualize high-dimensional data structures for intuitive user understanding, currently within research experiments about interactive anonymization. Rounding off his academic curiosity, Bernd is always interested in packaging his insights into usable, web-based, Software.


%----------------------------------------------------------------------------------------

%\bibliographystyle{apacite}
\bibliography{references}

\end{document} 
